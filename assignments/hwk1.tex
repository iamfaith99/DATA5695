% Options for packages loaded elsewhere
\PassOptionsToPackage{unicode}{hyperref}
\PassOptionsToPackage{hyphens}{url}
\PassOptionsToPackage{dvipsnames,svgnames,x11names}{xcolor}
%
\documentclass[
  letterpaper,
  DIV=11,
  numbers=noendperiod]{scrartcl}

\usepackage{amsmath,amssymb}
\usepackage{iftex}
\ifPDFTeX
  \usepackage[T1]{fontenc}
  \usepackage[utf8]{inputenc}
  \usepackage{textcomp} % provide euro and other symbols
\else % if luatex or xetex
  \usepackage{unicode-math}
  \defaultfontfeatures{Scale=MatchLowercase}
  \defaultfontfeatures[\rmfamily]{Ligatures=TeX,Scale=1}
\fi
\usepackage{lmodern}
\ifPDFTeX\else  
    % xetex/luatex font selection
\fi
% Use upquote if available, for straight quotes in verbatim environments
\IfFileExists{upquote.sty}{\usepackage{upquote}}{}
\IfFileExists{microtype.sty}{% use microtype if available
  \usepackage[]{microtype}
  \UseMicrotypeSet[protrusion]{basicmath} % disable protrusion for tt fonts
}{}
\makeatletter
\@ifundefined{KOMAClassName}{% if non-KOMA class
  \IfFileExists{parskip.sty}{%
    \usepackage{parskip}
  }{% else
    \setlength{\parindent}{0pt}
    \setlength{\parskip}{6pt plus 2pt minus 1pt}}
}{% if KOMA class
  \KOMAoptions{parskip=half}}
\makeatother
\usepackage{xcolor}
\setlength{\emergencystretch}{3em} % prevent overfull lines
\setcounter{secnumdepth}{-\maxdimen} % remove section numbering
% Make \paragraph and \subparagraph free-standing
\ifx\paragraph\undefined\else
  \let\oldparagraph\paragraph
  \renewcommand{\paragraph}[1]{\oldparagraph{#1}\mbox{}}
\fi
\ifx\subparagraph\undefined\else
  \let\oldsubparagraph\subparagraph
  \renewcommand{\subparagraph}[1]{\oldsubparagraph{#1}\mbox{}}
\fi


\providecommand{\tightlist}{%
  \setlength{\itemsep}{0pt}\setlength{\parskip}{0pt}}\usepackage{longtable,booktabs,array}
\usepackage{calc} % for calculating minipage widths
% Correct order of tables after \paragraph or \subparagraph
\usepackage{etoolbox}
\makeatletter
\patchcmd\longtable{\par}{\if@noskipsec\mbox{}\fi\par}{}{}
\makeatother
% Allow footnotes in longtable head/foot
\IfFileExists{footnotehyper.sty}{\usepackage{footnotehyper}}{\usepackage{footnote}}
\makesavenoteenv{longtable}
\usepackage{graphicx}
\makeatletter
\def\maxwidth{\ifdim\Gin@nat@width>\linewidth\linewidth\else\Gin@nat@width\fi}
\def\maxheight{\ifdim\Gin@nat@height>\textheight\textheight\else\Gin@nat@height\fi}
\makeatother
% Scale images if necessary, so that they will not overflow the page
% margins by default, and it is still possible to overwrite the defaults
% using explicit options in \includegraphics[width, height, ...]{}
\setkeys{Gin}{width=\maxwidth,height=\maxheight,keepaspectratio}
% Set default figure placement to htbp
\makeatletter
\def\fps@figure{htbp}
\makeatother

\KOMAoption{captions}{tableheading}
\makeatletter
\makeatother
\makeatletter
\makeatother
\makeatletter
\@ifpackageloaded{caption}{}{\usepackage{caption}}
\AtBeginDocument{%
\ifdefined\contentsname
  \renewcommand*\contentsname{Table of contents}
\else
  \newcommand\contentsname{Table of contents}
\fi
\ifdefined\listfigurename
  \renewcommand*\listfigurename{List of Figures}
\else
  \newcommand\listfigurename{List of Figures}
\fi
\ifdefined\listtablename
  \renewcommand*\listtablename{List of Tables}
\else
  \newcommand\listtablename{List of Tables}
\fi
\ifdefined\figurename
  \renewcommand*\figurename{Figure}
\else
  \newcommand\figurename{Figure}
\fi
\ifdefined\tablename
  \renewcommand*\tablename{Table}
\else
  \newcommand\tablename{Table}
\fi
}
\@ifpackageloaded{float}{}{\usepackage{float}}
\floatstyle{ruled}
\@ifundefined{c@chapter}{\newfloat{codelisting}{h}{lop}}{\newfloat{codelisting}{h}{lop}[chapter]}
\floatname{codelisting}{Listing}
\newcommand*\listoflistings{\listof{codelisting}{List of Listings}}
\makeatother
\makeatletter
\@ifpackageloaded{caption}{}{\usepackage{caption}}
\@ifpackageloaded{subcaption}{}{\usepackage{subcaption}}
\makeatother
\makeatletter
\@ifpackageloaded{tcolorbox}{}{\usepackage[skins,breakable]{tcolorbox}}
\makeatother
\makeatletter
\@ifundefined{shadecolor}{\definecolor{shadecolor}{rgb}{.97, .97, .97}}
\makeatother
\makeatletter
\makeatother
\makeatletter
\makeatother
\ifLuaTeX
  \usepackage{selnolig}  % disable illegal ligatures
\fi
\IfFileExists{bookmark.sty}{\usepackage{bookmark}}{\usepackage{hyperref}}
\IfFileExists{xurl.sty}{\usepackage{xurl}}{} % add URL line breaks if available
\urlstyle{same} % disable monospaced font for URLs
\hypersetup{
  pdftitle={Binomial Option Pricing},
  pdfauthor={Dr.~Tyler J. Brough},
  colorlinks=true,
  linkcolor={blue},
  filecolor={Maroon},
  citecolor={Blue},
  urlcolor={Blue},
  pdfcreator={LaTeX via pandoc}}

\title{Binomial Option Pricing}
\usepackage{etoolbox}
\makeatletter
\providecommand{\subtitle}[1]{% add subtitle to \maketitle
  \apptocmd{\@title}{\par {\large #1 \par}}{}{}
}
\makeatother
\subtitle{DATA 5695: Homework \#1}
\author{Dr.~Tyler J. Brough}
\date{2024-02-06}

\begin{document}
\maketitle
\ifdefined\Shaded\renewenvironment{Shaded}{\begin{tcolorbox}[boxrule=0pt, enhanced, frame hidden, interior hidden, breakable, borderline west={3pt}{0pt}{shadecolor}, sharp corners]}{\end{tcolorbox}}\fi

\section{Introduction}\label{introduction}

This homework assignment is all about the Binomial options pricing
model. It is based on Chapters 9, 10, 11 of the McDonald text.

\subsection{Chapter 9: Parity and Other Option
Relationships}\label{chapter-9-parity-and-other-option-relationships}

\textbf{9.1} A stock currently sells for \(\$32.00\). A \(6\)-month call
option with a strike of \(\$35\) has a premium of \(\$2.27\). Assuming a
\(4\%\) continuously compounded risk-free rate and a \(6\%\) continuous
dividend yield, what is the price of the associated put option?

\textbf{9.2.} A stock currently sells for \(\$32.00\). A \(6\)-month
call option with a strike of \(\$30.00\) has a premium of \(\$4.29\),
and a \(6\)-month put with the same strike has a premium of \(\$2.64\).
Assume a \(4\%\) continuously compounded risk-free rate. What is the
present value of the dividends payable over the next \(6\) months?

\textbf{9.3} Suppose the S\&R index is \(800\), the continuously
compounded risk-free rate is \(5\%\), and the dividend yield is \(0\%\).
A \(1\)-year \(815\)-strike European call costs \(\$75\) and a
\(1\)-year \(815\)-strike European put costs \(\$45\). Consider the
strategy of buying the stock, selling the \(815\)-strike call, and
buying the \(815\)-strike put.

\begin{itemize}
\item
  \textbf{a.} What is the rate of return on this position held until the
  expiration of the options?
\item
  \textbf{b.} What is the arbitrage implied by your answer to (a)?
\item
  \textbf{c.} What difference between the call and put prices would
  eliminate arbitrage?
\item
  \textbf{d.} What difference between the call and put prices eliminates
  arbitrage for strike prices of \(\$780\), \(\$820\), and \(\$840\)?
\end{itemize}

\textbf{9.8} Suppose call and put prices are given by

\begin{longtable}[]{@{}lrr@{}}
\toprule\noalign{}
\textbf{Strike} & \textbf{\(50\)} & \textbf{\(55\)} \\
\midrule\noalign{}
\endhead
\bottomrule\noalign{}
\endlastfoot
Call Premium & 9 & 10 \\
Put Premium & 7 & 6 \\
\end{longtable}

What no-arbitrage property is violated? What spread position would you
use to effect arbitrage? Demonstrate that the spread position is an
arbitrage?

\textbf{9.9} Suppose call and put prices are given by

\begin{longtable}[]{@{}lrr@{}}
\toprule\noalign{}
\textbf{Strike} & \textbf{\(50\)} & \textbf{\(55\)} \\
\midrule\noalign{}
\endhead
\bottomrule\noalign{}
\endlastfoot
Call premium & 16 & 10 \\
Put premium & 7 & 14 \\
\end{longtable}

What no-arbitrage property is violated? What spread position would you
use to effect arbitrage? Demonstrate that the spread position is an
arbitrage?

\textbf{9.10} Suppose call and put prices are given by

\begin{longtable}[]{@{}lrrr@{}}
\toprule\noalign{}
\textbf{Strike} & \textbf{\(50\)} & \textbf{\(55\)} & \textbf{\(60\)} \\
\midrule\noalign{}
\endhead
\bottomrule\noalign{}
\endlastfoot
Call premium & 18 & 14 & 9.50 \\
Put premium & 7 & 10.75 & 14.45 \\
\end{longtable}

Find the convexity violations. What spread would you use to effect
arbitrage? Demonstrate that the spread position is an arbitrage?

\textbf{9.12} In each case identify the arbitrage and demonstrate how
you would make money by creating a table showing your payoff.

\begin{itemize}
\item
  \textbf{a.} Consider two European options on the same stock with the
  same time to expiration. The \(90\)-strike call costs \(\$10\) and the
  \(\95\)-strike call costs \(\$4\).
\item
  \textbf{b.} Now suppose these options have \(2\) years to expiration
  and the continously compounded interest rate is \(10\%\). The
  \(90\)-strike call costs \(\$10\) and the \(95\)-strike call costs
  \(\$5.25\). Show again that there is an arbitrage opportunity.
  (\textbf{Hint:} It is important in this case that the options are
  European.)
\item
  \textbf{c.} Suppose that a \(90\)-strike European call sells for
  \(\$15\), a \(100\)-strike call sells for \(\$10\), and a
  \(105\)-strike calls sells for \(\$6\). Show how you could use an
  asymmetric butterfly to profit from this arbitrage opportunity.
\end{itemize}

\subsection{Chapter 10: Binomial Option Pricing -- Basic
Concepts}\label{chapter-10-binomial-option-pricing-basic-concepts}

\textbf{10.1} Let \(S = \$100\), \(K = \$105\), \(r = 8\%\),
\(T = 0.5\), and \(\delta = 0\). Let \(u = 1.3\), \(d = 0.8\), and
\(n = 1\).

\begin{itemize}
\item
  \textbf{a.} What are the premium, \(\Delta\), and \(B\) for a European
  call?
\item
  \textbf{b.} What are the premium, \(\Delta\), and \(B\) for a European
  put?
\end{itemize}

\textbf{10.2} Let \(S = \$100\), \(K = \$95\), \(r = 8\%\), \(T = 0.5\),
and \(\delta = 0\). Let \(u = 1.3\), \(d = 0.8\), and \(n = 1\).

\begin{itemize}
\item
  \textbf{a.} Verify that the price of a European call is \(\$16.196\).
\item
  \textbf{b.} Suppose you observe a call price of \(\$17\). What is the
  arbitrage?
\item
  \textbf{c.} Suppose you observe a call price of \(\$15.50\). What is
  the arbitrage?
\end{itemize}

\textbf{10.3} Let \(S = \$100\), \(K = \$95\), \(r = 8\%\), \(T = 0.5\),
and \(\delta = 0\). Let \(u = 1.3\), and \(d = 0.8\), and \(n = 1\).

\begin{itemize}
\item
  \textbf{a.} Verify that the price of a European put is \(\$7.471\).
\item
  \textbf{b.} Suppose you observe a put price of \(\$8\). What is the
  arbitrage?
\item
  \textbf{c.} Suppose you observe a put price of \(\$6\). What is the
  arbitrage?
\end{itemize}

\textbf{10.4} Obtain at least \(5\) years' worth of daily stock price
data for a stock of your choice.

\begin{itemize}
\item
  \textbf{1.} Compute annual volatility using all the data.
\item
  \textbf{2.} Compute annual volatility for each calendar year in your
  data. How does volatility vary over time?
\item
  \textbf{3.} Compute annual volatility for the first and second half of
  each year in your data. How much variation is there in your estimate?
\end{itemize}

\textbf{10.5} Obtain at least \(10\) years of daily data for five
stocks. Estimate annual volatility for each year for each asset in your
data. What do you observe about the pattern of historical volatility
over time? Does historical volatility move in tandem for different
assets?

\textbf{10.6} Let \(S = \$100\), \(K = \$95\), \(\sigma = 30\%\),
\(r = 8\%\), \(T = 1\), and \(\delta = 0\). Let \(u = 1.3\),
\(d = 0.8\), and \(n = 2\). Construct the binomial tree for a call
option. At each node provide the premium, \(\Delta\), and \(B\).

\textbf{10.8} Let \(S = \$100\), \(K = \$95\), \(\sigma = 30\%\),
\(r = 8\%\), \(T = 1\), and \(\delta = 0\). Let \(u = 1.3\),
\(d = 0.8\), and \(n = 2\). Construct the binomial tree for a European
put option. At each node provide the premium, \(\Delta\), and \(B\).

\textbf{10.10} Let \(S = \$100\), \(K = \$95\), \(\sigma = 30\%\),
\(r = 8\%\), \(T = 1\), and \(\delta = 0\). Let \(u = 1.3\),
\(d = 0.8\), and \(n = 2\). Construct the binomial tree for an American
put option. At each node provide the premium, \(\Delta\), and \(B\).

\textbf{10.11} Suppose \(S_{0} = \$100\), \(K = \$50\), \(r = 7.696\%\)
(continuously compounded), \(\delta = 0\), and \(T = 1\).

\begin{itemize}
\item
  \textbf{a.} Suppose that for \(h = 1\), we have \(u = 1.2\) and
  \(d = 1.05\). What is the binomial option price for a call option that
  lives one period? Is there any problem with having \(d > 1\)?
\item
  \textbf{b.} Suppose now that \(u = 1.4\) and \(d = 0.6\). Before
  computing the option price, what is your guess about how it will
  change from your previous answer? Does it change? How do you account
  for the result? Interpret your answer using put-call parity.
\item
  \textbf{c.} Now let \(u = 1.4\) and \(d = 0.4\). How do you think the
  call option price will change from (a)? How does it change? How do you
  account for this? Use put-call parity to explain your answer.
\end{itemize}

\textbf{10.12} Let \(S = \$100\), \(K = \$95\), \(r = 8\%\) (continously
compounded), \(\sigma = 30\%\), \(\delta = 0\), \(T = 1\) year, and
\(n = 3\).

\begin{itemize}
\item
  \textbf{a.} Verify that the binomial option price for an American call
  option is \(\$18.283\). Verify that there is never early exercise;
  hence, a European call would have the same price.
\item
  \textbf{b.} Show that the binomial option price for European put
  option is \(\$5.979\). Verify that put-call parity is satisfied.
\item
  \textbf{c.} Verify that the price of an American put is \(\$6.678\).
\end{itemize}

\textbf{10.21} For a stock index, \(S = \$100\), \(\sigma = 30\%\),
\(r = 5\%\), \(\delta = 3\%\), and \(T = 3\). Let \(n = 3\).

\begin{itemize}
\item
  \textbf{a.} What is the price of a European call option with a strike
  of \(\$95\)?
\item
  \textbf{b.} What is the price of a European put option with a strike
  of \(\$95\).
\item
  \textbf{c.} Now let \(S = \$95\), \(K = \$100\), \(\sigma = 30\%\),
  \(r = 3\%\), and \(\delta = 5\%\). (You have exchanged values for the
  stock price and strike price and for the interest rate and dividend
  yield.) Value both options again. What do you notice?
\end{itemize}

\textbf{10.22} Repeat the previous problem calculating prices for
American options instead of European. What happens?

\subsection{Chapter 11: Binomial Option Pricing -- Selected
Topics}\label{chapter-11-binomial-option-pricing-selected-topics}

For the following problems write three functions in Python:
\texttt{call\_payoff}, \texttt{put\_payoff}, and \texttt{binomial}. The
first two functions are to calculate the payoff to plain vanilla call
and put options. The third function is to calculate the option premium,
\(\Delta\), and \(B\) for a given pricing problem. Use these functions
for any of the problems below for which \(n \gt 3\).

\textbf{11.1} Consider a one-period binomial model with \(h = 1\), where
\(S = \$100\), \(r = 0\), \(\sigma = 30\%\), and \(\delta = 0.08\).
Compute American call options for \(K = \$70, \$80, \$90\), and
\(\$100\).

\begin{itemize}
\item
  \textbf{a.} At which strike(s) does early exercise occur?
\item
  \textbf{b.} Use put-call parity to explain why early exercise does not
  occur at the higher strikes.
\item
  \textbf{c.} Use put-call parity to explain why early exercise is sure
  to occur for all lower strikes than that in your answer to (a).
\end{itemize}

\textbf{11.4} Consider a one-period binomial model with \(h = 1\), where
\(S = \$100\), \(r = 0.08\), \(\sigma = 30\%\), and \(\delta = 0\).
Compute American put option prices for \(K = \$100, \$110, \$120\), and
\(\$130\).

\begin{itemize}
\item
  \textbf{a.} At which strike(s) does early exercise occur?
\item
  \textbf{b.} Use put-call parity to explain why early exercise is sure
  to occur for all strikes greater than that in your answer to (a).
\end{itemize}

\textbf{11.5} Repeat Problem 11.4, only set \(\delta = 0.08\). What is
the lowest strike price at which early exercise will occur? What
condition related to put-call parity is satisfied at this strike price?

\textbf{11.7} Let \(S = \$100\), \(K = \$100\), \(\sigma = 30\%\),
\(r = 0.08\), \(t = 1\), and \(\delta = 0\). Let \(n = 10\). Suppose the
stock has an expected return of \(15\%\).

\begin{itemize}
\item
  \textbf{a.} What is the expected return on a European call option? A
  European put option?
\item
  \textbf{b.} What happens to the expected return if you increase the
  volatility to \(50\%\)?
\end{itemize}

\textbf{11.8} Let \(S = \$100\), \(\sigma = 30\%\), \(r = 0.08\),
\(t = 1\), and \(\delta = 0\). Suppose the true expected return on the
stock is \(15\%\). Set \(n = 10\). Compute European call prices,
\(\Delta\), and \(B\) for strikes of
\(\$70, \$80, \$90, \$100, \$110, \$120\), and \(\$130\). For each
strike, compute the expected return on the option. What effect does the
strike have on the option's expected return?

\textbf{11.9} Repeat the previous problem, except that for each strike
price, compute the expected return on the option for times to expiration
of \(3\) months, \(6\) months, \(1\) year, and \(2\) years. What effect
does time to maturity have on the option's expected return?

\textbf{11.10} Let \(S = \$100\), \(\sigma = 30\%\), \(r = 0.08\),
\(t = 1\), and \(\delta = 0\). Suppose the true expected return on the
stock is \(15\%\). Set \(n = 10\). Compute European put prices,
\(\Delta\), and \(B\) for strikes of
\(\$70, \$80, \$90, \$100, \$110, \$120\), and \(\130\). For each
strike, compute the expected return on the option. What effect does the
strike have on the option's expected return?

\textbf{11.11} Repeat the previous problem, except that for each strike
price, compute the expected return on the option for times to expiration
of \(3\) months, \$6\% months, \(1\) year, and \(2\) years. What effect
does time to maturity have on the option's expected return?

\textbf{11.12} Let \(S = \$100\), \(\sigma = 30\%\), \(r = 0.08\),
\(t = 1\), and \(\delta = 0\). Using equation (11.12) to compute the
probability of reaching a terminal node and \(Su^{i}d^{n-i}\) to compute
the price at that node, plot the risk-neutral distribution of year-1
stock prices as in Figures 11.7 and 11.8 for \(n = 3\) and \(n = 10\).

\textbf{11.13} Repeat the previous problem for \(n = 50\). What is the
risk-neutral probability that \(S_{1} < \$80\)? \(S_{1} < \$120\)?

\textbf{11.14} We saw in Section 10.1 that the undiscounted risk-neutral
expected stock price equals the forward price. We will verify this using
the binomial tree in Figure 11.4.

\begin{itemize}
\item
  \textbf{a.} Using \(S = \$100\), \(r = 0.08\), and \$\delta = 0.08,
  what are the \(4\)-month, \(8\)-month, and \(1\)-year forward prices?
\item
  \textbf{b} Verify your answers in (a) by computing the risk-neutral
  expected stock price in the first, second, and third binomial period.
  Use equation (11.12) to determine the probability of reaching each
  node.
\end{itemize}

\subsection{Bonus Problems}\label{bonus-problems}

\textbf{1} Write a Python function to calculate the implied volatility
using the binomial options pricing model. Obtain 1-year of historical
daily stock price data and compute historical and implied volatilities
using your function for at-the-money strike prices. Plot and compare the
results.

\textbf{2} Let \(S = \$41\), \(\sigma = 30\%\), \(r = 0.08\), \(t = 1\),
and \(\delta = 0\). Simulate the delta-hedging process of the options
market-maker on a daily basis for a written \(40\)-strike call option.
Assume that the true expected rate of return on the stock is
\(\mu = 15\%\).

\begin{itemize}
\item
  \textbf{a.} Using the lognormal expressions for
  \(u = e^{(\mu - \delta - 0.5\sigma^{2})h + \sigma \sqrt{h}}\) and
  \(d = e^{(\mu - \delta - 0.5\sigma^{2})h - \sigma \sqrt{h}}\) and the
  binomial process for the stock price data-generating mechanism.
\item
  \textbf{b.} Using geometric Brownian motion for the stock price
  data-generating mechanism.
\item
  \textbf{c.} Using the stationary bootstrap and 1-year of history data
  as your data-genterating mechanism.
\end{itemize}

Compare the respective end-of-period hedging payoffs in dollar terms.



\end{document}
